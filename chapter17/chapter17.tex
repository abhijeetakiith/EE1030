% \iffalse
\let\negmedspace\undefined
\let\negthickspace\undefined
\documentclass[article,12pt,onecolumn]{IEEEtran}
\usepackage{cite}
\usepackage{amsmath,amssymb,amsfonts,amsthm}
\usepackage{algorithmic}
\usepackage{graphicx}
\usepackage{textcomp}
\usepackage{xcolor}
\usepackage{txfonts}
\usepackage{listings}
\usepackage{enumitem}
\usepackage{mathtools}
\usepackage{gensymb}
\usepackage{multicol}
\usepackage{comment}
\usepackage[breaklinks=true]{hyperref}
\usepackage{tkz-euclide} 
\usepackage{listings}
\usepackage{gvv}                                    
\def\inputGnumericTable{}                        
\usepackage[latin1]{inputenc}                     
\usepackage{color}                                  
\usepackage{array}                                  
\usepackage{longtable}                             
\usepackage{calc}                                   
\usepackage{multirow}                               
\usepackage{hhline}                                 
\usepackage{ifthen}           
\usepackage{lscape}
\newtheorem{theorem}{Theorem}[section]
\newtheorem{problem}{Problem}
\newtheorem{proposition}{Proposition}[section]
\newtheorem{lemma}{Lemma}[section]
\newtheorem{corollary}[theorem]{Corollary}
\newtheorem{example}{Example}[section]
\newtheorem{definition}[problem]{Definition}
\newcommand{\BEQA}{\begin{eqnarray}}
\newcommand{\EEQA}{\end{eqnarray}}
\newcommand{\define}{\stackrel{\triangle}{=}}
\theoremstyle{remark}
\newtheorem{rem}{Remark}

% Marks the beginning of the document
\begin{document}
\bibliographystyle{IEEEtran}
\vspace{3cm}
\title{CH - 17 INDEFINITE INTEGRALS}
\author{AI24BTECH11001 - Abhijeet Kumar}
\maketitle
\bigskip
\section{A : JEE Advance / IIT-JEE}
\renewcommand{\thefigure}{\theenumi}
\renewcommand{\thetable}{\theenumi}
\subsection{Fill in the Blanks}
\begin{enumerate}
\item If 
\begin{align*}
    \int\frac{4e^{x}+ 6e^{-x}}{9e^{x} + 4e^{-x}} = Ax+B\log\brak{9e^{2x}-4} + C
\end{align*}
Then A=\dotso,B=\dotso and C=\dotso
\hfill \brak{1990-2 Marks}
\end{enumerate}
\subsection{MCQs with One Correct Answer}
\begin{enumerate}
\item The value of the integral  
\begin{align*}
    \int\frac{\cos^{3}x + \cos^{5}x}{\sin^{2}x + \sin^{4}x}dx
\end{align*} 
is
\hfill \brak{1995S}
\begin{enumerate}
    \item $\sin x - 6\tan^{-1}\brak{\sin x}+ c$
    \item $\sin x - 2 \brak{\sin x}^{-1} + c$
    \item $\sin x - 2 \brak{\sin x}^{-1} - 6 \tan^{-1}\brak{\sin x} + c$ 
    \item $\sin x - 2 \brak{\sin x}^{-1} - 6 \tan^{-1}\brak{\sin x} + c$
\end{enumerate}
\item If 
\begin{align*}
    \int_{\sin x}^{1}t^2 
 f\brak{t}dt = 1- \sin x,  
\end{align*} 
then $f\brak{\frac{1}{\sqrt{3}}}$ is 
\hfill \brak{2005S}
\begin{multicols}{2}
\begin{enumerate}
    \item $\frac{1}{3}$
    \item $\frac{1}{\sqrt{3}}$
    \item $3$ 
    \item $\sqrt{3}$
\end{enumerate}
\end{multicols}
\item 
\begin{align*}
    \int\frac{x^{2} - 1}{x^{3}\sqrt{2x^{4}+2x^{2}+1}}dx = 
\end{align*}
\hfill \brak{2006 -3M,-1}
\begin{enumerate}
    \item $\frac{\sqrt{2x^{4}-2x^{2}+1}}{x^{2}} + c$
    \item $\frac{\sqrt{2x^{4}-2x^{2}+1}}{x^{3}} +c$
    \item $\frac{\sqrt{2x^{4}-2x^{2}+1}}{x} +c$ 
    \item $\frac{\sqrt{2x^{4}-2x^{2}+1}}{2x^{2}} +c$
\end{enumerate}
\item Let 
\begin{align*}
    I=\int\frac{e^{x}}{e^{4x}+e^{2x} +1}dx , 
 J=\int\frac{e^{-x}}{e^{-4x}+e^{-2x} +1}dx, 
\end{align*} Then for an arbitrary constant C,the value of J - I equals 

\hfill\brak{2008}
\begin{multicols}{2}
\begin{enumerate}
    \item $\frac{1}{2}\log\brak{\frac{e^{4x}-e^{2x}+1}{e^{4x}+e^{2x}+1}} + C$
    \item $\frac{1}{2}\log\brak{\frac{e^{2x}+e^{x}+1}{e^{2x}-e^{x}+1}} + C$
    \item $\frac{1}{2}\log\brak{\frac{e^{2x}-e^{x}+1}{e^{2x}+e^{x}+1}} + C$ 
    \item $\frac{1}{2}\log\brak{\frac{e^{4x}+e^{2x}+1}{e^{4x}-e^{2x}+1}} + C$
\end{enumerate}
\end{multicols}
\item The Integral $\int\frac{\sec^{2}x}{\brak{\sec x +\tan x}^\frac{9}{2}}dx $  equals (for some arbitrary constant $K$)
\hfill \brak{2012}
\begin{enumerate}
    \item $-\frac{1}{\brak{\sec x + \tan x}^{\frac{11}{2}}} \{ \frac{1}{11} - \frac{1}{7}\brak{\sec x +\tan x}^{2}\} + K$
    \item $\frac{1}{\brak{\sec x + \tan x}^{\frac{11}{2}}} \{ \frac{1}{11} - \frac{1}{7}\brak{\sec x +\tan x}^{2}\} + K$
    \item $-\frac{1}{\brak{\sec x + \tan x}^{\frac{11}{2}}} \{ \frac{1}{11} + \frac{1}{7}\brak{\sec x +\tan x}^{2}\} + K$
    \item  $\frac{1}{\brak{\sec x + \tan x}^{\frac{11}{2}}} \{ \frac{1}{11} + \frac{1}{7}\brak{\sec x +\tan x}^{2}\} + K$
\end{enumerate}
\end{enumerate}
\subsection{Subjective Problems}    
\begin{enumerate}
\item Evaluate $\int \frac{\sin x}{\sin x - \cos x}dx $
\hfill\brak{1978}
\item Evaluate $\int \frac{x^{2}}{\brak{a+bx}^{2}}dx$
\hfill \brak{1979}
\item Evaluate $\int \brak{e^{\log x} + \sin x}cos xdx $
\hfill \brak{1981 - 2Marks}
\item Evaluate $\int \frac{\brak{x-1}e^{x}}{\brak{x+1}^{3}} dx$
\hfill \brak{1983 - 2Marks}
\end{enumerate}
\end{document}