\let\negmedspace\undefined
\let\negthickspace\undefined
\documentclass[journal]{IEEEtran}
\usepackage[a5paper, margin=10mm, onecolumn]{geometry}
%\usepackage{lmodern} % Ensure lmodern is loaded for pdflatex
\usepackage{tfrupee} % Include tfrupee package

\setlength{\headheight}{1cm} % Set the height of the header box
\setlength{\headsep}{0mm}     % Set the distance between the header box and the top of the text

\usepackage{gvv-book}
\usepackage{gvv}
\usepackage{cite}
\usepackage{amsmath,amssymb,amsfonts,amsthm}
\usepackage{algorithmic}
\usepackage{graphicx}
\usepackage{textcomp}
\usepackage{xcolor}
\usepackage{txfonts}
\usepackage{listings}
\usepackage{enumitem}
\usepackage{mathtools}
\usepackage{gensymb}
\usepackage{comment}
\usepackage[breaklinks=true]{hyperref}
\usepackage{tkz-euclide} 
\usepackage{listings}
\usepackage{amsmath}
% \usepackage{gvv}                                        
\def\inputGnumericTable{}                                 
\usepackage[latin1]{inputenc}                                
\usepackage{color}                                            
\usepackage{array}                                            
\usepackage{longtable}                                       
\usepackage{calc}  
\usepackage{circuitikz}
\usepackage{multirow}                                         
\usepackage{hhline}                                           
\usepackage{ifthen}                                           
\usepackage{lscape}
\usepackage{tikz}
\begin{document}

\bibliographystyle{IEEEtran}
\vspace{3cm}


\renewcommand{\thefigure}{\theenumi}
\renewcommand{\thetable}{\theenumi}
\setlength{\intextsep}{10pt} % Space between text and floats


\numberwithin{equation}{enumi}
\numberwithin{figure}{enumi}
\renewcommand{\thetable}{\theenumi}

\title{GATE XE - 2015}
\author{AI24BTECH11001 Abhijeet Kumar
}
\maketitle
\renewcommand{\thefigure}{\theenumi}
\renewcommand{\thetable}{\theenumi}
\begin{enumerate}[start=27]
    \item An open glass capillary tube of $2 mm$ bore is lowered into a cistern containing mercury (density = $13600 kg/m^3$) as shown in the figure. Given that the contact angle between mercury and glass = $140\circ$, surface tension coefficient = $0.484 N/m$ and gravitational acceleration = $9.81 m/s^2$, the depression of mercury in the capillary tube below the free surface in the cistern, in mm, is

    \begin{figure}[H]
    \centering
    \resizebox{0.5\textwidth}{!}{%
    \begin{circuitikz}
    \tikzstyle{every node}=[font=\Huge]
    \draw[domain=0.5:3.75,samples=100,smooth, line width=0.8pt] plot (\x,{0.6*sin(1*\x r -0.5 r ) +4.5});
    \draw [line width=0.8pt, short] (3.5,17) -- (3.75,3.5);
    \draw [line width=0.8pt, short] (0.25,16.75) -- (0.5,3.25);
    \draw [line width=0.8pt, short] (-8.5,12.25) -- (-8,0.25);
    \draw [line width=0.8pt, short] (-8,0.25) -- (12.5,0.25);
    \draw [line width=0.8pt, short] (11.75,12.25) -- (12.25,0.25);
    \draw [line width=0.8pt, short] (-8.25,10) -- (0.5,10);
    \draw [line width=0.8pt, short] (3.75,10) -- (12,10);
    \draw [short] (-8.25,8.75) -- (-6,10);
    \draw [short] (-8.25,7.75) -- (-4.75,9.75);
    \draw [short] (-8.25,6.75) -- (-3,10);
    \draw [short] (-8.25,5.75) -- (-2,9.75);
    \draw [short] (-8,4.75) -- (-0.25,10);
    \draw [short] (-8,3.5) -- (0.25,9.25);
    \draw [short] (-8,2.25) -- (0.5,8.25);
    \draw [short] (-8,1.25) -- (0.5,7);
    \draw [short] (-7.75,0.25) -- (0.5,5.75);
    \draw [short] (3.75,9) -- (5.25,10);
    \draw [short] (3.75,7.75) -- (6.75,10);
    \draw [short] (-5.5,0.25) -- (1.75,5);
    \draw [short] (3.75,6.25) -- (9,10);
    \draw [short] (-3.5,0.25) -- (3.25,4.5);
    \draw [short] (3.75,4.75) -- (11,10);
    \draw [short] (-1.25,0.25) -- (12,9);
    \draw [short] (1.25,0.5) -- (12,7.5);
    \draw [short] (3,0.25) -- (12,6);
    \draw [short] (5.25,0.25) -- (12,4.5);
    \draw [short] (7.5,0.25) -- (12.25,3.25);
    \draw [short] (9.75,0.25) -- (12.25,1.75);
    \draw [->, >=Stealth] (5.25,14) -- (3.75,12.75);
    \draw [->, >=Stealth] (-2,12) -- (-0.75,8.5);
    \draw [line width=0.7pt, short] (-11.5,5) -- (2.75,5.25);
    \draw [<->, >=Stealth] (-10.25,10) -- (-10,5);
    \draw [short] (-11.5,10) -- (-9.5,10);
    \draw [short] (-2.75,12) -- (-2,12);
    \draw [->, >=Stealth] (2.75,10.5) -- (3,5.25);
    \draw [short] (2.75,10.5) -- (4.75,11.25);
    \node [font=\Huge] at (-4.5,12) {\textbf{Mercury}};
    \node [font=\Huge] at (1.75,6.75) {\textbf{Air}};
    \node [font=\Huge] at (6.25,11.25) {\textbf{Meniscus}};
    \node [font=\Huge] at (-12.5,7.5) {\textbf{Depression}};
    \node [font=\Huge] at (6.5,16.25) {\textbf{Glass}};
    \node [font=\Huge] at (6.5,15.5) {\textbf{capillary}};
    \node [font=\Huge] at (6.5,14.75) {\textbf{tube}};
    \end{circuitikz}
    }%

    \label{fig:my_label}
    \end{figure}

    \item Consider a combined forced-free vortex. The central region with radius $R$ and angular velocity $\omega$ is the forced vortex and the rest is the free vortex. The pressure at the edge of the combined vortex is $p_o$. If the density of the fluid is $p$, the pressure at the center of the combined vortex is
    \begin{multicols}{4}
        \begin{enumerate}
            \item $p_o - p\omega^2 R^2$
            \item $p_o - \frac{1}{2}p\omega^2 R^2$
            \item $p_o + \frac{1}{2}p\omega^2 R^2$
            \item $p_o + p\omega^2 R^2$
        \end{enumerate}
    \end{multicols}

    \item Which one of the following is true at the point of separation of a boundary layer.
    \begin{enumerate}
        \item Transition occurs from laminar to turbulent flow
        \item The flow relaminarizes from turbulent regime
        \item The shear stress vanishes
        \item The relation between stress and rate of strain ceases to be linear
    \end{enumerate}

    \item A certain fluid flow phenomenon is described by Reynolds, Weber and Ohnesorge numbers. The Weber and Ohnesorge numbers are defined as $We = \frac{pU^2L}{\sigma}$ and $Oh =\frac{\mu}{\sqrt{p\sigma L}} $ respectively, where $\sigma$ is the surface tension, $p$ is the density, $\mu$ is the dynamic viscosity, $U$ is the velocity and $L$ is the characteristic dimension. If $Re$ denotes the Reynolds number, which of the following relations is true?
    \begin{multicols}{4}
        \begin{enumerate}
            \item $We=Oh Re^2$
            \item $We=Oh^2 Re^2$
            \item $We=Oh^2 Re$
            \item $We=Oh Re$
        \end{enumerate}
    \end{multicols}

    \item A rectangular boat $6$ m wide and $15$ m long (dimension perpendicular to the plane of the figure) has a draught of $2$ m. The side view of the boat is as shown in the figure. The centre of gravity $G$ of the boat is at the free surface level. The metacentric height of the boat in m is 
    \begin{figure}[H]
    \centering
    \resizebox{0.5\textwidth}{!}{%
    \begin{circuitikz}
    \tikzstyle{every node}=[font=\Large]
    \draw  (-0.5,8) rectangle (6.75,3.75);
    \draw [short] (-1.5,6.25) -- (8,6.25);
    \draw [dashed] (3,8.5) -- (3,3);
    \draw [short] (-0.5,9) -- (-0.5,8.5);
    \draw [short] (6.75,9) -- (6.75,8.5);
    \draw [<->, >=Stealth] (-0.5,8.75) -- (6.75,8.75);
    \draw [<->, >=Stealth] (-1.25,6.25) -- (-1.25,3.75);
    \draw [short] (-1.5,3.75) -- (-1,3.75);
    \node [font=\Large] at (2.75,9) {Width = 6m};
    \node [font=\Large] at (-2.25,5.25) {Draught};
    \node [font=\Large] at (-2.5,4.75) {= 2 m};
    \node [font=\Large] at (2.75,6.5) {G};
    \draw [short] (7.25,6.75) -- (7.5,6.25);
    \draw [short] (7.75,6.75) -- (7.5,6.25);
    \draw [short] (7.25,6.75) -- (7.75,6.75);
    \draw (7.5,6.25) to (7.5,6) node[ground]{};
    \end{circuitikz}
    }%

    \label{fig:my_label}
    \end{figure}
    \begin{multicols}{4}
        \begin{enumerate}
            \item $-1.0$
            \item $0.5$
            \item $1.5$
            \item $2.0$
        \end{enumerate}
    \end{multicols}

    \item Water drains out into atmosphere from a small orifice located at the bottom of a large open tank. If the initial height of the water column is $H$, the time taken to empty the tank is proportional to
    \begin{multicols}{4}
        \begin{enumerate}
            \item $H^{\frac{1}{2}}$
            \item $H$
            \item $H^{\frac{3}{2}}$
            \item $H^{2}$
        \end{enumerate}
    \end{multicols}

    \item Consider a two-dimensional velocity field given by $\overrightarrow{V} = \pi y \hat{i} - \pi x \hat{j}$ where $\hat{i}$ and $\hat{j}$ are the unit vectors in the directions of the rectangular Cartesian coordinates $x$ and $y$, respectively. A fluid particle is located initially at the point $\brak{-1,1}$. Its position after unit time is
    \begin{multicols}{4}
        \begin{enumerate}
            \item $\brak{-2,-2}$
            \item $\brak{1,-1}$
            \item $\brak{1,1}$
            \item $\brak{3,-1}$
        \end{enumerate}
    \end{multicols}

    \item The figure shows a reducing area conduit carrying water. The pressure $p$ and velocity $V$ are uniform across sections $1$ and $2$. The density of water is $1000$ kg/$m^3$. If the total loss of head due to friction is just equal to the loss of potential head between the inlet and the outlet, then $V_2$ in $m/s$ will be

    \begin{figure}[H]
    \centering
    \resizebox{0.5\textwidth}{!}{%
    \begin{circuitikz}
    \tikzstyle{every node}=[font=\large]
    \begin{scope}[rotate around={-40.25:(-3.5,12.75)}]
    \draw[domain=-3.5:3,samples=100,smooth] plot (\x,{0.7*sin(1*\x r +3.5 r ) +12.75});
    \end{scope}
    \begin{scope}[rotate around={-34.25:(-4.25,11.5)}]
    \draw[domain=-4.25:2.5,samples=100,smooth] plot (\x,    {0.7*sin(1*\x r +3.75 r ) +11.5});
    \end{scope}
    \draw [dashed] (-4,13.25) -- (-4,10.75);
    \draw [short] (-4.5,12.75) -- (-3.5,12.75);
    \draw [short] (-4.75,11.75) -- (-4,11.75);
    \draw [dashed] (-4.5,12.25) -- (2,12);
    \draw [->, >=Stealth] (-5.75,12.25) -- (-4,12.25);
    \draw [<->, >=Stealth] (0.75,12) -- (0.75,8.75);
    \draw [->, >=Stealth] (1,8.5) -- (2.5,8.5);
    \draw [dashed] (0.75,8.75) -- (0.75,8);
    \node [font=\Large] at (1,10.5) {$\textbf{z}$};
    \node [font=\large] at (-4,10.25) {$\textbf{Inlet}$};
    \node [font=\large] at (-4,10.75) {$\textbf{1}$};
    \node [font=\large] at (0.75,7.75) {$\textbf{2}$};
    \node [font=\large] at (0.75,7.25) {$\textbf{Outlet}$};
    \node [font=\large] at (-6.5,12.5) {$\textbf{$V_1 = 2 m/s$}$};
    \node [font=\large] at (-6.75,12) {$\textbf{$p_1 = 130  kPa$}$};
    \node [font=\large] at (3,8.5) {$\textbf{$V_2$}$};
    \node [font=\large] at (3,8) {$\textbf{$p_2 = 100 kPa$}$};
    \end{circuitikz}
    }%

    \label{fig:my_label}
    \end{figure}

    \item A fluid enters a control volume through an inlet port (denoted with subscript $'i'$) and leaves through two outlet ports (denoted with subscripts $'o,1'$ and $'o,2'$) as shown in the figure. The velocities may be assumed to be uniform across the ports. The rate of change of mass in the control volume in $kg/s$, at the instant shown in the figure, is

    \begin{figure}[H]
    \centering
    \resizebox{0.5\textwidth}{!}{%
    \begin{circuitikz}
    \tikzstyle{every node}=[font=\small]
    
    \draw (3.25,11.5) to[short] (8.75,11.5);
    \draw (8.75,11.5) to[short] (8.75,8);
    \draw (3.25,11.5) to[short] (3.25,10.25);
    \node at (3.25,10.25) [circ] {};
    \draw [dashed] (3.25,10.25) -- (3.25,8.75);
    \node at (3.25,8.75) [circ] {};
    \node at (8.75,8) [circ] {};
    \draw (3.25,8.75) to[short] (3.25,6.75);
    \node at (3.25,6.75) [circ] {};
    \draw [dashed] (8.75,8) -- (8.75,7.25);
    \draw [dashed] (3.25,6.75) -- (3.25,5.75);
    \draw (3.25,5.75) to[short] (8.75,5.75);
    \node at (8.75,7.5) [circ] {};
    \draw (8.75,7.5) to[short] (8.75,5.75);
    \draw [dashed] (8.75,7.75) -- (11,7.75);
    \draw (8.75,8) to[short] (8.25,8);
    \draw (8.75,7.5) to[short] (8.25,7.5);
    \draw [<->, >=Stealth] (8.5,8) -- (8.5,7.5);
    \draw [->, >=Stealth] (8.75,7.75) -- (10.75,8.25);
    \draw [->, >=Stealth] (9.5,8.5) -- (9.75,8);
    \draw [->, >=Stealth] (10.25,7.25) -- (10.25,7.75);
    \draw [dashed] (3.25,9.5) -- (1,9.5);
    \draw [short] (3.25,9.5) -- (2.5,10);
    \draw [->, >=Stealth] (1.25,10.75) -- (2.5,10);
    \draw [dashed] (3.25,6.25) -- (1,6.25);
    \draw [->, >=Stealth] (3.25,6.25) -- (1,7);
    \draw [short] (3.25,10.25) -- (4,10.25);
    \draw [short] (3.25,8.75) -- (4,8.75);
    \draw [<->, >=Stealth] (3.75,10.25) -- (3.75,8.75);
    \draw [short] (3.25,6.75) -- (4,6.75);
    \draw [<->, >=Stealth] (3.75,6.75) -- (3.75,5.75);
    \node [font=\small] at (1.5,11) {$\rho=5.5 \, \frac{\text{kg}}{\text{m}^{3}}$};
    \node [font=\small] at (2,10.5) {$V_i=3.0 \,    \frac{\text{m}}{\text{s}}$};
    \node [font=\small] at (2,9.75) {$40^\circ$};
    \node [font=\small] at (2,9.25) {Inlet};
    \node [font=\small] at (1.25,7.75) {$\rho_{o,1}=5.0 \, \frac{\text{kg}}{\text{m}^{3}}$};
    \node [font=\small] at (2,7.25) {$v_{o,1}=2.0 \, \frac{\text{m}}{\text{s}}$};
    \node [font=\small] at (2,6.5) {$30^\circ$};
    \node [font=\small] at (2,6) {Outlet-1};
    \node [font=\small] at (4.5,9.5) {$A_i=0.20 \, \text{m}^{2}$};
    \node [font=\small] at (10,9) {$\rho_{o,2}=5.0 \, \frac{\text{kg}}{\text{m}^{3}}$};
    \node [font=\small] at (10,8.5) {$v_{o,2}=1.0 \, \frac{\text{m}}{\text{s}}$};
    \node [font=\small] at (10.25,8) {$20^\circ$};
    \node [font=\small] at (10.5,7.25) {Outlet-2};
    \node [font=\small] at (7.5,7.75) {$A_{o,1}=0.10 \, \text{m}^{2}$};
    \node [font=\small] at (4.75,6.25) {$A_{i,1}=0.15 \, \text{m}^{2}$};
    \end{circuitikz}
    }%
    \label{fig:my_label}
    \end{figure}

    \item The total discharge of water through a lawn sprinkler shown in the figure is one liter per minute. The velocity of the jet at each end, relative to the arm, is $27\pi m/s$. The density of water is $1000 kg/m^3$ and the length of each arm is $0.1 m$. If the frictional torque of the pivot is $\pi/36 mN.m$, the rotational speed, in revolutions per minute, of the sprinkler is
    \begin{figure}[H]
    \centering
    \resizebox{0.5\textwidth}{!}{%
    \begin{circuitikz}
    \tikzstyle{every node}=[font=\LARGE]
    \draw  (3.5,6.25) circle (2.25cm);
    \draw  (3.5,6.25) circle (0.25cm);
    \draw [short] (-0.5,6.5) -- (8,6.5);
    \draw [short] (-1,6) -- (7.5,6);
    \draw [short] (-1,7.75) -- (-1,6);
    \draw [short] (-0.5,7.75) -- (-0.5,6.5);
    \draw [short] (7.5,6) -- (7.5,4.5);
    \draw [short] (8,6.5) -- (8,4.5);
    \draw [->, >=Stealth] (6,6.25) -- (7.25,6.25);
    \draw [->, >=Stealth] (7.75,5.75) -- (7.75,4.75);
    \draw [->, >=Stealth] (-0.75,6.75) -- (-0.75,7.75);
    \draw [->, >=Stealth] (0.75,6.25) -- (-0.25,6.25);
    \draw [->, >=Stealth] (1.75,4.75) -- (2.5,5.5);
    \draw [short] (1.75,4.75) -- (0.5,4.75);
    \draw [->, >=Stealth] (5.5,8.75) -- (4.75,8);
    \draw [->, >=Stealth] (3.75,8.5) -- (3.5,8.5);
    \draw [->, >=Stealth] (3.5,4) -- (3.75,4);
    \node [font=\Large] at (-0.25,4.75) {Pivot};
    \node [font=\Large] at (6.5,9.5) {Direction};
    \node [font=\Large] at (6.5,9) {of rotation};
    \end{circuitikz}
    }%

    \label{fig:my_label}
    \end{figure}

    \item Consider a two-dimensional potential flow field with the radial and tangential velocity components, $v_{r} = \frac{m}{2\pi r}$ and $v_{\theta} = \frac{k}{2\pi r}$ respectively, where $m$ and $k$ are constants. The stream function is such that it increases along the direction of traverse of a line in the flow field if the flow is from left to right across that line. The stream function $\Psi$ for this flow field, with $\psi = 0$ at  $r = a$ and $\theta = 0$ , is
    \begin{multicols}{2}
        \begin{enumerate}
            \item $\frac{m \pi \theta}{2} + \frac{k \pi}{2} \log{\frac{r}{a}}$
            \item $\frac{m \pi \theta}{2} - \frac{k \pi}{2} \log{\frac{r}{a}}$
            \item $\frac{m  \theta}{2\pi} + \frac{k }{2\pi} \log{\frac{r}{a}}$
            \item $\frac{m  \theta}{2\pi} - \frac{k }{2\pi} \log{\frac{r}{a}}$
        \end{enumerate}
        
    \end{multicols}

    \item A steady, two-dimensional, inviscid and incompressible flow field is described in rectangular Cartesian coordinates as $u = ax$ and $v=-ay$, where $u$ and $v$ are the components of the velocity vector in the $x$ and $y$ directions, respectively. Gravity acts along the negative y-direction. The pressure distribution, with the reference pressure taken as zero at the origin, with usual notation, is given by
    \begin{multicols}{2}
        \begin{enumerate}
            \item $-\frac{1}{2}p a^2 \brak{x^2 +xy +y^2} - pgy$
            \item $-\frac{1}{2}p a^2 \brak{x^2 -xy +y^2} - pgy$
            \item $-\frac{1}{2}p a^2 \brak{x^2 +y^2} - pgy$
            \item $-\frac{1}{2}p a^2 \brak{x^2 - y^2} - pgy$
        \end{enumerate}
    \end{multicols}

    
    \item A cylinder of radius $0.1 m$ rotating clockwise about its own axis at an angular velocity of $100/\pi$ radians per second is placed in a cross-stream of air flowing at a velocity of $10 m/s$ from left to right. The density of air is $1.2 kg/m^3$. The lift force per unit length of the cylinder in $N/m$ is
\end{enumerate}
\end{document}
